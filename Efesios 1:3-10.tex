\documentclass[11pt,letterpaper]{article}
\usepackage[utf8]{inputenc}
\usepackage{amsmath}
\usepackage{amsfonts}
\usepackage{amssymb}
\usepackage{graphicx}
\usepackage[width=15.00cm, height=15.00cm, left=1.00cm, right=1.00cm, top=1.00cm, bottom=1.00cm]{geometry}
\author{Ciencias}
\begin{document}
	\centering {\Large {ANÁLISIS DE PÁRRAFOS}}\\
	{\large EFESIOS Versículos 1:3-10}\\
	
	\begin{itemize}
		\item {\large Oraciones presentes en el texto.}
		\begin{enumerate}
			\item Alabado sea Dios, Padre de nuestro Señor Jesucristo, que nos ha bendecido en las regiones celestiales con toda bendición espiritual en Cristo.
			\item Dios nos escogió en él antes de la creación del mundo, para que seamos santos y sin mancha delante de él.
			\item En amor nos predestinó para ser adoptados como hijos suyos por medio de Jesucristo, según el buen propósito de su voluntad, para alabanza de su gloriosa gracia, que nos concedió en su Amado.
			\item En él tenemos la redención mediante su sangre, el perdón de nuestros pecados, conforme a las riquezas de la gracia que Dios nos dio en abundancia con toda sabiduría y entendimiento.
			\item Él nos hizo conocer el misterio de su voluntad conforme al buen propósito que de antemano estableció en Cristo, para llevarlo a cabo cuando se cumpliera el tiempo, esto es, reunir en él todas las cosas, tanto las del cielo como las de la tierra.
		\end{enumerate}
		\item {\large Clápsulas independientes y subordinadas.} 
		\begin{enumerate}
			\item Alabado sea Dios, Padre de nuestro Señor Jesucristo, que nos ha bendecido en las regiones celestiales con toda bendición espiritual en Cristo
			\begin{enumerate}
				\item Alabado sea Dios
				\item Padre de nuestro Señor Jesucristo
				\item que nos ha bendecido en las regiones celestiales con toda bendición espiritual en Cristo
			\end{enumerate}
			\item Dios nos escogió en él antes de la creación del mundo, para que seamos santos y sin mancha delante de él.
			\begin{enumerate}
				\item Dios nos escogió en él antes de la creación del mundo
				\item para que seamos santos
				\item y sin mancha delante de él.
			\end{enumerate}
		\end{enumerate}
	\end{itemize}

	
	
\end{document}